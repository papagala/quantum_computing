%{{{ Definimos parámetros iniciales
\documentclass{beamer}
\usepackage[spanish]{babel}
%\input{header2.tex}
\usepackage[utf8]{inputenc}
\usepackage{color}
\usetheme{Warsaw}
\usepackage{ragged2e} 

\newcommand{\ket}[1]{\left| #1 \right>} % for Dirac bras
\newcommand{\bra}[1]{\left< #1 \right|} % for Dirac kets

\title[Quantum Computing]{An introduction}
\author{Oswaldo Gomez}
\institute{AI Engineering}
\date{\today}
\begin{document}
%}}}
%Start of slide
\begin{frame}{Quantum computing}
\titlepage
\end{frame}

%Start of slide
\begin{frame}{Basic Unit of information}
\justifying
Traditional computation works with $0$ and 1 as 

\end{frame}

%Start of slide
\begin{frame}{Computational basis states}
\justifying
Qubits can be in different states \textit{other} than $\ket{0}$ or $\ket{1}$. It is possible to form \textit{linear combinations} of states, called superpositions:
$$\ket{\psi}= \alpha \ket{0} + \beta \ket{1}$$
The numbers $\alpha$ and $\beta$ are complex numbers and ${|\alpha|}^2+|\beta|^2=1$.  
\end{frame}

\begin{frame}{Observaciones}
Si únicamente se invierte en el activo riesgoso, el proceso Xt tiene caídas de mayor magnitud que si se combinan inversión y reaseguro e incluso sólo reaseguro.
\newline
\newline
Razón:  Al invertir en un activo riesgoso estamos sujetos a la volatilidad de este.
\end{frame}




\begin{frame}{Bibliografía}
\begin{itemize}
\item Glasserman, Paul; Monte Carlo Methods in Financial Engineering, Springer-Verlag, New York 2004.
\newline
\item Glynn, Peter W. Asmussen Soren; Stochastic simulation, algorithms and analysis;Springer Science+Bussines Media, 2007
\newline
\item Hanspeter, Schmidli; Asymptotics of ruin probabilities for risk processes under optimal reinsurance policies: the small claim case.
\end{itemize}






\end{frame}

\end {document}



